
\definecolor{KU-red}{RGB}{144,26,30}
\mdfdefinestyle{leftrule}{%
 skipabove=12,skipbelow=12pt,
 innertopmargin=0.4em,%
 innerbottommargin=0.4em,%
 innerrightmargin=0.7em,%
 rightmargin=0.7em,%
 innerleftmargin=1.7em,%
 leftmargin=0.7em,%
 linecolor=KU-red,%
 backgroundcolor=KU-red!15,%
 linewidth=2pt,
 topline=false,
 bottomline=false,
 rightline=false,
 leftline=true
}


\usepackage{xparse}

\ExplSyntaxOn
\NewDocumentCommand{\pair}{sO{}m}
{
    \IfBooleanTF{#1}
    {\left(\coord_print:n {#3}\right)}
    {\mathopen{#2(}\coord_print:n {#3}\mathclose{#2)}}
}
\seq_new:N \l_coord_list_seq
\tl_new:N \l_coord_last_tl
\cs_new_protected:Npn \coord_print:n #1
{
\seq_set_split:Nnn \l_coord_list_seq { , } { #1 }
\seq_pop_right:NN \l_coord_list_seq \l_coord_last_tl
\seq_map_inline:Nn \l_coord_list_seq { ##1 , }
\tl_use:N \l_coord_last_tl
}
\ExplSyntaxOff


\mdtheorem[style=leftrule,backgroundcolor=yellow!15,linecolor=yellow!90]{definition}{Definition}
\mdtheorem[style=leftrule]{example}{Example}
\mdtheorem[style=leftrule]{exercise}{Exercise}
\mdtheorem[style=leftrule,backgroundcolor=blue!15,linecolor=blue!90]{theorem}{Theorem}
\mdtheorem[style=leftrule,backgroundcolor=gray!15,linecolor=gray!90]{proof}{Proof}


\renewcommand{\matrix}[1]{\mathbf{#1}} 
\DeclareMathOperator\Span{span}
\DeclareMathOperator\proj{proj}
\DeclareMathOperator\col{col}
\DeclareMathOperator\Null{null}
\DeclareMathOperator\prik{\bullet}
\newcommand{\into}{\rightarrow}
\newcommand{\row}[1]{\mathbf{r}_{#1}} 
\newcommand{\norm}[1]{||#1||} 
\newcommand{\sfrac}[2]{^{#1}\!/\!_{#2}} 
\newcommand{\wline}{\begin{center}\rule{\linewidth}{\linethickness}\end{center}}
\newcommand{\set}[1]{\left\{ #1 \right\}}
%\newcommand{\exp}[1]{\mathrm{e}^{#1}}
\newcommand{\R}{\mathbb{R}}
\newcommand{\transpose}{\top}
\newcommand{\N}{\mathbb{N}}
\newcommand{\Q}{\mathbb{Q}}
\newcommand{\Z}{\mathbb{Z}}
\newcommand{\calP}{\mathcal{P}}
\newcommand{\E}[1]{\cdot10^{#1}}
